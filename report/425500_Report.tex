\documentclass[12pt,oneside]{book} % for one-sided printing

\usepackage{blindtext}% Just used so we can generate some example text
\usepackage{amsmath}
\usepackage{algorithm}
\usepackage{algpseudocode}
\usepackage{amssymb}
\usepackage{mathtools}
\usepackage[export]{adjustbox}
\usepackage{lipsum}
\usepackage{booktabs}  % For better quality tables
\usepackage{tabularx}  % for the X column type
\usepackage{listings}
\usepackage{xcolor}
\usepackage{caption}
\usepackage{xfrac}
\usepackage{indentfirst}
\usepackage{subcaption}
\usepackage{graphicx}
\usepackage{geometry}
\geometry{a4paper, margin=1in}

% Place style file after other packages.
\usepackage{cranfieldthesis}
\usepackage{lscape} % for landscape pages
\usepackage{float}
\usepackage[toc,title,page]{appendix}

% Couleurs personnalisées
\definecolor{backcolour}{rgb}{0.96, 0.96, 0.96} % Fond très clair
\definecolor{codegray}{rgb}{0.47, 0.47, 0.47}   % Commentaires et numéros de ligne
\definecolor{codegreen}{rgb}{0.25, 0.50, 0.35}  % Commentaires
\definecolor{codeblue}{rgb}{0.26, 0.44, 0.58}   % Mots-clés
\definecolor{codepurple}{rgb}{0.50, 0, 0.50}    % Identificateurs
\definecolor{codeteal}{rgb}{0, 0.5, 0.5}        % Chaînes de caractères
\definecolor{terminalback}{rgb}{0.05, 0.05, 0.05} % Fond très sombre pour le terminal
\definecolor{terminaltext}{rgb}{0.7, 0.7, 0.7}    % Texte clair pour le terminal
\definecolor{mygreen}{rgb}{0,0.6,0}
\definecolor{mygray}{rgb}{0.5,0.5,0.5}
\definecolor{mymauve}{rgb}{0.58,0,0.82}
\definecolor{terminalbgcolor}{HTML}{330033}
\definecolor{terminalrulecolor}{HTML}{000099}

\lstdefinestyle{pythonstyle}{
    language=Python,
    backgroundcolor=\color{backcolour},
    commentstyle=\color{codegreen},
    keywordstyle=\color{codeblue},
    numberstyle=\tiny\color{codegray},
    stringstyle=\color{codeteal},
    identifierstyle=\color{codepurple},
    basicstyle=\ttfamily\scriptsize,
    breakatwhitespace=false,
    breaklines=true,
    captionpos=b,
    keepspaces=true,
    numbers=left,
    numbersep=5pt,
    showspaces=false,
    showstringspaces=false,
    showtabs=false,
    tabsize=4,
    frame=single,
    rulecolor=\color{codegray},
    framexleftmargin=15pt,
    framextopmargin=5pt,
    framexbottommargin=5pt,
    framexrightmargin=15pt,
}

\lstdefinestyle{bashstyle}{
    language=bash,
    backgroundcolor=\color{backcolour},
    basicstyle=\ttfamily\scriptsize,
    keywordstyle=\color{blue},
    stringstyle=\color{red},
    identifierstyle=\color{codepurple},
    commentstyle=\color{codegreen},
    morecomment=[l]{\#},   % Define comment style
    frame=single,          % adds a frame around the code
    rulecolor=\color{gray},% if not set, the frame-color may be changed on line-breaks
    breakatwhitespace=false,
    breaklines=true,       % sets automatic line breaking
    captionpos=b,          % sets the caption-position to bottom
    keepspaces=true,       % keeps spaces in text
    showspaces=false,      % show spaces everywhere adding particular underscores
    showstringspaces=false % underline spaces within strings only
}

\lstdefinestyle{ymlstyle}{
    language=Python,
    backgroundcolor=\color{backcolour},
    basicstyle=\ttfamily\scriptsize,
    keywordstyle=\color{blue},
    stringstyle=\color{red},
    identifierstyle=\color{codepurple},
    commentstyle=\color{codegreen},
    morecomment=[l]{\#},   % Define comment style
    frame=single,          % adds a frame around the code
    rulecolor=\color{gray},% if not set, the frame-color may be changed on line-breaks
    breakatwhitespace=false,
    breaklines=true,       % sets automatic line breaking
    captionpos=b,          % sets the caption-position to bottom
    keepspaces=true,       % keeps spaces in text
    showspaces=false,      % show spaces everywhere adding particular underscores
    showstringspaces=false, % underline spaces within strings only
    literate=% Special handling for dot notations
        {http.response_time.p99}{{\textcolor{blue}{http.response\_time.p99}}}{18}
        {http.response_time.p95}{{\textcolor{blue}{http.response\_time.p95}}}{18}
        {metrics-by-endpoint}{{\textcolor{blue}{metrics-by-endpoint}}}{16},
    morekeywords={config, flow, target, phases, plugins, scenarios, loop, get, url, count, ensure, apdex, threshold, thresholds, duration, arrivalRate, name, pause, metrics -by-endpoint}, % Additional YAML-specific keywords
}

\lstdefinestyle{servicestyle}{
backgroundcolor=\color{backcolour},      % choose the background color
basicstyle=\ttfamily\scriptsize, % print whole listing white
keywordstyle=\color{blue},          % attributes in blue
stringstyle=\color{red},            % values in red
identifierstyle=\color{codepurple},
breakatwhitespace=false,
breaklines=true,                    % sets automatic line breaking
captionpos=b,                       % sets the caption-position to bottom
keepspaces=true,                    % keeps spaces in text
showspaces=false,                   % show spaces everywhere adding particular underscores
showstringspaces=false,             % underline spaces within strings only
frame=single,                       % adds a frame around the code
rulecolor=\color{gray},             % if not set, the frame-color may be changed on line-breaks
morecomment=[l]{\#},                % Define comment style
moredelim=[s][\color{codegreen}]{[}{]}, % color everything between [ and ] in green
morekeywords={Description,Wants,After,ExecStart,WantedBy,EnvironmentFile,User,Group,Type,Restart,Documentation,WorkingDirectory,RuntimeDirectory,RuntimeDirectoryMode,LimitNOFILE,TimeoutStopSec,CapabilityBoundingSet,DeviceAllow,LockPersonality,MemoryDenyWriteExecute,NoNewPrivileges,PrivateDevices,PrivateTmp,ProtectClock,ProtectControlGroups,ProtectHome,ProtectHostname,ProtectKernelLogs,ProtectKernelModules,ProtectKernelTunables,ProtectProc,ProtectSystem,RemoveIPC,RestrictAddressFamilies,RestrictNamespaces,RestrictRealtime,RestrictSUIDSGID,SystemCallArchitectures,UMask} % add your attributes here
}

% Title Page Set Up
\title{High Performance Technical Computing Assignment}
\author{Alexis Balayre}
\date{5\textsuperscript{th} February 2024}
\school{\SATM}
\degree{MSc}
\course{Computational Software of Techniques Engineering}
\academicyear{2023 - 2024}

% Supervisors
\supervisor{Dr Irene Moulitsas}

% Copyright
\copyrightyear{2024}

% References
\usepackage[numbers]{natbib} % for nice referencing
\makeatletter % Reference list option change to number and period
\renewcommand\@biblabel[1]{#1.} % from [1] to 1
\makeatother %
\setcitestyle{round} % use round citations

\begin{document}

\frontmatter

% Form Title Pages
\maketitle

% Abstract and Keywords
\begin{abstract}
    Replace with your abstract text of not more than 300 words.
\end{abstract}

% Acknowledgements
\chapter{Acknowledgements}
The author would like to thank \dots

% Use single spacing for Table of Contents, List of Figures, etc
{
    \clearpage
    \singlespacing
    % Table of Contents
    {
        \tableofcontents
    }
    \clearpage

    % List of Figures
    \listoffigures

    % List of Tables
    \listoftables
}

%% Main Matter
\mainmatter
\pagestyle{fancy}
\fancyhead[L]{\nouppercase{\leftmark}}
\fancyhead[R]{\nouppercase{\rightmark}}

\chapter{Introduction}

\chapter{Methodology}
\section{Sequential Algorithm}

The algorithm for multiplying a sparse matrix by a dense vector can be
efficiently implemented using the Compressed Sparse Row (CSR) format. The CSR
format represents a sparse matrix using three arrays: \texttt{values},
\texttt{col\_indices}, and \texttt{row\_pointers}. Given a sparse matrix \( M
\) in CSR format and a dense vector \( v \), the product \( M \times v \) is
computed as follows:

\begin{algorithm}[H]
    \caption{Sequential algorithm}
    \begin{algorithmic}[1]
        \Procedure{SparseMatrixVectorMult}{$M$, $v$}
        \State Initialise a result vector $r$ of appropriate size
        \For{each row $i$ of $M$}
        \State Determine the start and end of row $i$ in $M$
        \For{each non-zero element $M_{ij}$ in row $i$}
        \State Find the corresponding column index $j$ in $M$
        \State Compute the product of $M_{ij}$ and $v_j$
        \State Add the product to the $i$-th element of $r$
        \EndFor
        \EndFor
        \State \textbf{return} $r$
        \EndProcedure
    \end{algorithmic}
\end{algorithm}

\subsection{Complexity Analysis}

The complexity analysis of the algorithm for multiplying a sparse matrix by a
dense vector focuses on two main aspects: temporal complexity and spatial
complexity.

\subsubsection{Temporal Complexity}

The temporal complexity of the algorithm depends on how the sparse matrix is
stored and the number of non-zero elements in the matrix.

\begin{itemize}
    \item \textbf{Traversing Rows:} The algorithm traverses each row of the matrix. If the matrix has \( m \) rows, this step has a complexity of \( O(m) \).
    \item \textbf{Traversing Non-Zero Elements:} Inside each row, the algorithm traverses the non-zero elements. If the total number of non-zero elements in the matrix is \( n_{nz} \), the traversal of all these elements has a complexity of \( O(n_{nz}) \).
\end{itemize}

The total temporal complexity is therefore \( O(m + n_{nz}) \). However, in
practice, this complexity is often considered as \( O(n_{nz}) \), as the number
of non-zero elements is usually the dominating factor, especially in very
sparse matrices.

\subsubsection{Spatial Complexity}

The spatial complexity is related to the amount of memory required by the
algorithm.

\begin{itemize}
    \item \textbf{Storing the Sparse Matrix:} The way the sparse matrix is stored affects the spatial complexity. Generally, storage formats like CSR or COO allow storing a sparse matrix with a complexity of \( O(n_{nz}) \), where \( n_{nz} \) is the number of non-zero elements.
    \item \textbf{Dense Vector:} The dense vector has a spatial complexity of \( O(n) \), where \( n \) is the size of the vector.
    \item \textbf{Result Vector:} The result vector also has a size of \( O(m) \), where \( m \) is the number of rows in the matrix.
\end{itemize}

The total spatial complexity is therefore \( O(n_{nz} + m + n) \), but in
practice, the focus is mainly on the \( O(n_{nz}) \) term as it is generally
the most significant.

\subsection{Example}

Dans le format CSR, la matrice est représentée par trois vecteurs :
\texttt{values}, \texttt{rows}, et \texttt{cols}. Pour notre exemple, ces
vecteurs sont définis comme suit:
\begin{itemize}
    \item \texttt{values} = \{1, 2, 3, 4\}
    \item \texttt{rows} = \{0, 2, 3, 3, 4\}
    \item \texttt{cols} = \{0, 2, 1, 3\}
\end{itemize}

La matrice creuse correspondante peut être visualisée comme:
\[
    \begin{bmatrix}
        1 & 0 & 2 & 0 \\
        0 & 0 & 3 & 0 \\
        0 & 0 & 0 & 0 \\
        0 & 0 & 0 & 4 \\
    \end{bmatrix}
\]

Le vecteur dense est simplementq:
\[
    \begin{bmatrix}
        1 \\
        2 \\
        3 \\
        4 \\
    \end{bmatrix}
\]

La multiplication de la matrice creuse par le vecteur dense est effectuée ligne
par ligne. Le résultat peut être visualisé comme :

\[
    \begin{bmatrix}
        1 & 0 & 2 & 0 \\
        0 & 0 & 3 & 0 \\
        0 & 0 & 0 & 0 \\
        0 & 0 & 0 & 4 \\
    \end{bmatrix}
    \times
    \begin{bmatrix}
        1 \\
        2 \\
        3 \\
        4 \\
    \end{bmatrix}
    =
    \begin{bmatrix}
        1 \times 1 + 0 \times 2 + 2 \times 3 + 0 \times 4 \\
        0 \times 1 + 0 \times 2 + 3 \times 3 + 0 \times 4 \\
        0 \times 1 + 0 \times 2 + 0 \times 3 + 0 \times 4 \\
        0 \times 1 + 0 \times 2 + 0 \times 3 + 4 \times 4 \\
    \end{bmatrix}
    =
    \begin{bmatrix}
        7  \\
        9  \\
        0  \\
        16 \\
    \end{bmatrix}
\]

\section{Line-Based Parallelism}
This algorithm divides the sparse matrix into chunks of rows, distributing
these chunks across multiple threads or processes.

\begin{algorithm}[H]
    \caption{Line-based parallel sparse matrix-vector multiplication}
    \begin{algorithmic}[1]
        \Procedure{ParallelLineBasedMult}{$M$, $v$, $numThreads$}
        \State Divide $M$ into $numThreads$ row chunks
        \State Initialise a result vector $r$ of appropriate size
        \For{each thread $t$}
        \State Assign a row chunk to thread $t$
        \State Thread $t$ computes multiplication for its chunk
        \State Accumulate results in $r$
        \EndFor
        \State \textbf{return} $r$
        \EndProcedure
    \end{algorithmic}
\end{algorithm}

\subsection{Complexity Analysis}
\subsubsection{Temporal Complexity}
\begin{itemize}
    \item \textbf{Dividing the Matrix into Chunks:} The complexity of dividing the matrix \(M\) into \texttt{numThreads} chunks is \(O(m)\), where \(m\) is the number of rows in the matrix. This step involves assigning row ranges to each thread and is relatively lightweight in terms of computation.
    
    \item \textbf{Multiplication within Each Thread:} Each thread processes its assigned chunk of the matrix. Assuming \(n_{nz}\) is the total number of non-zero elements in the matrix and the elements are evenly distributed, the complexity for this part is approximately \(O\left(\frac{n_{nz}}{\texttt{numThreads}}\right)\) in an ideal parallel environment.
    
    \item \textbf{Accumulation of Results:} The final step of accumulating the results in the vector \(r\) can be executed in parallel and does not significantly add to the complexity, assuming ideal conditions.
\end{itemize}

\subsubsection{Spatial Complexity}
The spatial complexity is influenced by the storage requirements for the sparse matrix and the result vector.
\begin{itemize}
    \item \textbf{Sparse Matrix Storage:} Storing the sparse matrix in CSR (Compressed Sparse Row) format requires \(O(n_{nz} + m + n)\) space, where \(n_{nz}\) is the number of non-zero elements, \(m\) is the number of rows, and \(n\) is the number of columns.
    
    \item \textbf{Result Vector Storage:} The result vector \(r\) of size \(m\) has a spatial complexity of \(O(m)\).
    
    \item \textbf{Temporary Storage in Threads:} Additional temporary storage for computations within each thread does not significantly affect the overall spatial complexity.
\end{itemize}

\subsubsection{Overall Complexity}
\begin{itemize}
    \item \textbf{Temporal Complexity:} The overall temporal complexity of the algorithm is \(O(m + \frac{n_{nz}}{\texttt{numThreads}})\), under the assumption of an ideal parallel environment.
    
    \item \textbf{Spatial Complexity:} The overall spatial complexity is \(O(n_{nz} + m + n)\), dominated by the storage requirements for the sparse matrix and the result vector.
\end{itemize}


\subsection*{2. Non-Zero Element-Based Parallelism}

This approach distributes the computation based on the non-zero elements of the
sparse matrix.

\begin{algorithm}[H]
    \caption{Element-based parallel sparse matrix-vector multiplication}
    \begin{algorithmic}[1]
        \Procedure{ParallelElementBasedMult}{$M$, $v$, $numThreads$}
        \State Divide non-zero elements of $M$ into $numThreads$ parts
        \State Initialise a result vector $r$ of appropriate size
        \For{each thread $t$}
        \State Assign a subset of non-zero elements to thread $t$
        \State Thread $t$ computes contributions to $r$
        \State Perform atomic addition to accumulate in $r$
        \EndFor
        \State \textbf{return} $r$
        \EndProcedure
    \end{algorithmic}
\end{algorithm}

\subsection*{3. Hybrid Parallelism (Lines and Non-Zero Elements)}

This method combines the first two approaches: it divides the matrix into row
chunks and further parallelizes by distributing the non-zero elements of each
chunk.

\begin{algorithm}[H]
    \caption{Hybrid parallel sparse matrix-vector multiplication}
    \begin{algorithmic}[1]
        \Procedure{HybridParallelMult}{$M
            $, $v$, $numThreads$}
        \State Divide $M$ into row chunks
        \State Initialise a result vector $r$ of appropriate size
        \For{each row chunk}
        \State Further divide non-zero elements among threads
        \For{each thread $t$}
        \State Assign a subset of non-zero elements to thread $t$
        \State Thread $t$ computes contributions to $r$
        \State Perform atomic addition to accumulate in $r$
        \EndFor
        \EndFor
        \State \textbf{return} $r$
        \EndProcedure
    \end{algorithmic}
\end{algorithm}

\subsection*{Complexity and Considerations}

Each of these parallel algorithms aims to exploit different aspects of
parallelism, with the primary goal of reducing the overall computation time.
The actual performance gain depends on the characteristics of the sparse
matrix, the number of available processing units, and the specific
implementation details. Moreover, care must be taken to manage concurrency
issues, such as race conditions and proper synchronization, to ensure correct
and efficient execution.

\newpage
\chapter{Conclusion}

\bibliographystyle{CranfieldNumbered}
\bibliography{CUCitations}

\appendix
\chapter{Documentation}

\begin{subappendices}
    \section{Project tree}
    \begin{lstlisting}[breaklines=true, basicstyle=\small]
    lib/
        collecting.py
        processing.py
        storing.py
    scripts/
        get_iam_credentials.sh
        start_spark_job.sh
    services/
        get_iam_credentials.service
        spark_python_job.service
    test/
        artillery_load_test.yml
        monitoring.py
        metrics.csv
        results.json
        visualisation_load_test.ipynb
    main.py
    README.md
    requirements.txt
    \end{lstlisting}

    \section{Getting Started}
    To run the program, follow these steps:
    \begin{enumerate}
        \itemindent=17.87pt
        \item Create a virtual environment using \texttt{python3 -m venv venv}.
        \item Activate the virtual environment using \texttt{source venv/bin/activate}.
        \item Install the required dependencies using \texttt{pip3 install -r
                  requirements.txt}.
        \item Run the program using \texttt{python3 main.py}.
        \item Visualise the results using \texttt{visualisation.ipynb} (Jupyter Notebook).
    \end{enumerate}

    \section{Detailed Features of Functions}
    \begin{description}
        \item \texttt{collecting.py}
              \begin{itemize}
                  \item \texttt{fetch\_sensors\_data(sparkSession)}: Function to ingest the latest data from the sensors and returns it as a Spark DataFrame.
              \end{itemize}

        \item \texttt{processing.py}
              \begin{itemize}
                  \item \texttt{get\_aqi\_value\_p25(value)}: Function for calculating the AQI value for PM2.5.
                  \item \texttt{get\_aqi\_value\_p10(value)}: Function for calculating the AQI value for PM10.
                  \item \texttt{computeAQI(df)}: Function for calculating the AQI value for each particulate matter sensor and returning the DataFrame with the AQI column.
              \end{itemize}

        \item \texttt{storing.py}
              \begin{itemize}
                  \item \texttt{keepOnlyUpdatedRows(database\_name, table\_name, df)}: Function for keeping only the rows that have been updated in the DataFrame.
                  \item \texttt{\_print\_rejected\_records\_exceptions(err)}: Internal function for printing the rejected records exceptions.
                  \item \texttt{write\_records(database\_name, table\_name, client, records)}: Internal function for writing a batch of records to the Timestream database.
                  \item \texttt{writeToTimestream(database\_name, table\_name, partionned\_df)}: Function for writing the DataFrame to the Timestream database.
              \end{itemize}
    \end{description}
\end{subappendices}

\end{document}

